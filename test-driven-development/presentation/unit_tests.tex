\begin{frame}{Unit Tests: Idea}
Goal:
\begin{itemize}
\item Isolate each part of the program and show that the individual parts are correct
\item Allows refactoring of code at a later date, and makes sure the module is still working correctly
\end{itemize}
How:
\begin{itemize}
\item Many small independent and isolated tests
\item Automated tests run by developer
\item Used in TDD
\end{itemize}
\end{frame}

\begin{frame}{Unit Tests: Rules}
Unit tests should be:
\begin{description}
\item[small] e.g. concentrate on single criteria, so that failing of a test directly points to underlying error \\
Rule of thumb: Consider if the test is a logical AND between conditions or a logical OR . In the former case go for multiple assertions, in the latter create multiple test functions.
\item[fast] in order to run whole test set after each change/save to the code
\item[idempotent] e.g. should be order-independent and not rely on each other
\item[isolated] e.g. should be independent of environment or external APIs
\end{description}
\end{frame}

\begin{frame}{Test flows and types}
A component deals with several messages\footnote{Link: \url{https://speakerdeck.com/skmetz/magic-tricks-of-testing-railsconf?slide=38}}.
Messages can be differentiated into \textit{incoming}, \textit{outgoing} and \textit{private} messages and two different types (\textit{query} and \textit{command}).\\
\hfill\\
Message flows can be described by tuple \textit{(source, origin)}:
\begin{description}
\item[Incoming] (outside, self)
\item[Outgoing] (self, outside)
\item[Private] (self, self)
\end{description}
Message types:
\begin{description}
\item[Query] Return something, change nothing
\item[Command] Return nothing, change something
\end{description}
\end{frame}

\begin{frame}{Test flows and types}
How to test those 6 cases (3 flows x 2 types)?
\begin{description}
\item[Incoming query] \hfill\\
 Easy test; assert result of query
\item[Incoming command] \hfill\\
Call command and check changes by query
\item[Outgoing query/command] \hfill\\
External API not part of test, so mock external API and check only if query/command is sent
\item[Private query/command] \hfill\\
Do not test!\\
Implicitly tested by incoming/outgoing tests; test only if you think you should for some reason!
\end{description}
\end{frame}

\begin{frame}{Test flows and types: Summary}
\centering\begin{tabular}{ l c c}
  \textbf{Flow} & \textbf{Type} & \textbf{Test?} \\
  \hline
  Incoming & Query & Yes \\
  Incoming & Command & Yes \\
  Outgoing & Query & Mock \\
  Outgoing & Command & Mock \\
  Private & Query & No/Maybe \\
  Private & Command & No/Maybe \\
\end{tabular}
\end{frame}

\begin{frame}{Other Tests}
\begin{description}
\item[Integration Test] \hfill\\
Occurs after unit testing; combines multiple unit tests to a group and tests group outcome.\\
Example:\\
Login at GUI $\rightarrow$ API call $\rightarrow$ DB check $\rightarrow$ login
\item[Regression Test] \hfill \\
Re-running functional and non-functional tests to ensure that previously developed and tested software still performs after a change.
\item[Smoke Test] \hfill\\
Smoke tests are a subset of test cases that cover the most important functionality of a component or system, used to aid assessment of whether main functions of the software appear to work correctly.
\item[...]
\end{description}
\end{frame}